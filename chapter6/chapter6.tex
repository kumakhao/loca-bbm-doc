\chapter{Fazit und Ausblick}
\label{chap:FazitundAusblick}
Es konnte gezeigt werden, dass das entwickelte Verfahre, in der erstellten Simulation, eine Lokalisation auf der B�hne erlaubt. Die durchgef�hrten Versuche zeigen, den starken Einfluss der Trajektorie des Roboters auf der B�hne, insbesondere wenn nur wenige Bilder pro Minute ausgewertet werden k�nnen. Da Drehungen und Kurven die gr��te Quelle f�r Unsicherheiten sind haben sie den st�rksten Einfluss auf die gesch�tzte Position. und ... durch Personen.
Der Ansatz des Verfahrens, nicht auf den Daten der Bilder zu arbeiten, konnte in der Simulation erfolgreich umgesetzt werden. 

R�ckblickend h�tten zus�tzlich Versuche durchgef�hrt werden k�nnen, um den Einfluss verschiedene Qualit�ten des Bit-Musters zu untersuchen. Ferner h�tten Testreihen zum Verhalten des Filters bei verschieden gew�hlten Unsicherheiten in der Simulation durchgef�hrt werden k�nnen.


Um das Verfahren weiter zu entwickeln, k�nnten zum einen weite Versuche in der Simulation unternommen werden. Es k�nnte untersucht werden, wie gro� der Einfluss einer Ungenauen Kamerakalibrierung auf die Lokalisation ist. Eine Erweiterung der Simulation w�re auch denkbar, indem die Lichtwand-Texturen mit einem Helligkeitsgradienten versehen werden. Ferne w�re interessant die Lokalisation mit einem realen Roboter zu testen. Daf�r m�sste ein Verfahren f�r die Kamerakalibrierung am Roboter entwickelt werden. 