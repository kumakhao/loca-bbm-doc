% Dokumentart, Seitenformat
\documentclass[pdftex, a4paper, twoside, open=right, parskip, numbers=noenddot, listof=totoc, bibliography=totocnumbered]{scrreprt}

% Seitenabst�nde
\usepackage[bindingoffset=1cm, left=2.5cm, right=2.5cm, top=2.5cm, bottom=2.5cm]{geometry}

% Code for creating empty pages
% No headers on empty pages before new chapter
\makeatletter
\def\cleardoublepage{\clearpage\if@twoside \ifodd\c@page\else
    \hbox{}
    \thispagestyle{plain}
    \newpage
    \if@twocolumn\hbox{}\newpage\fi\fi\fi}
\makeatother \clearpage{\pagestyle{plain}\cleardoublepage}

% Kopfzeilen
\usepackage{fancyhdr}
\pagestyle{fancy}
\renewcommand{\chaptermark}[1]{\markboth{\thechapter\ #1}{}}
\fancyhead[LO,RE]{\leftmark}
\fancyhead[RO,LE]{\thepage}
\cfoot{}
\fancypagestyle{plain}{}

% Eingabe von �, �, �, � erlauben
\usepackage[ansinew]{inputenc}
\usepackage[T1]{fontenc}

% Deutsche Trennungen, Anf�hrungsstriche und mehr:
\usepackage[ngerman]{babel}
\usepackage[babel, german=quotes]{csquotes}

% Mathe
\usepackage{amsmath}

% Farben einbinden
\usepackage{color}
\usepackage{colortbl}

% Farben
\definecolor{tableBlue}{RGB}{79, 129, 189} % Tabellen
\definecolor{lightyellow}{rgb}{1, 1, 0.8}  % Quellcode


% Tabellen
\usepackage{tabularx}
\usepackage{multirow}


% Bilder einbinden
\usepackage{graphicx, wrapfig}
\usepackage{gnuplottex}

% Fix zur Bildereinbindung
\renewcommand{\textfraction}{0}


% Hyperlinks
\usepackage[hyphens]{url}
\usepackage{hyperref}
\hypersetup{colorlinks, citecolor=black, linkcolor=black, urlcolor=black}


% Quellcode einbinden
\usepackage{listings}
\renewcommand{\lstlistlistingname}{Quellcodeverzeichnis}
\lstset{aboveskip=\bigskipamount, backgroundcolor=\color{lightyellow}, basicstyle=\ttfamily\small, breaklines, breakautoindent, captionpos=b, columns=flexible, extendedchars, float=hbp, frame=single, keywordstyle=\bfseries, language=C, numbers=left, numberstyle=\tiny, showspaces=false, showstringspaces=false, showtabs=false, stringstyle=, tabsize=2}


% Abk�rzungsverzeichnis
\usepackage[printonlyused]{acronym}

% Stiel des Literaturverzeichnisses.
\bibliographystyle{apalike}
