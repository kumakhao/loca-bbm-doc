\chapter{Grundlagen}
\label{chap:grundlagen}
Die Lokalisation ist eines der Grundprobleme das beim Einsatz von mobilen Robotern auftritt. In \cite[Seite 193]{Thrun2006} wird die Lokalisation in drei Teilprobleme zerlegt:
\begin{description}
\item[Position Tracking] bezeichnet den Vorgang, bei dem mit bekannter Ausgangsposition diese mit Hilfe von Sensordaten bei Bewegungen verfolgt werden kann. Dabei spielt das Dynamikmodell des Roboters sowie darin modellierte Unsicherheiten eine wichtige Rolle. Denn bewegt sich der Roboter von einer bekannten Position aus, wird mit dem Dynamikmodell seine neue Position gesch�tzt. Bekannte Unsicherheiten des Modells erzeugen eine Wahrscheinlichkeitsverteilung um diese neue Position, in der sich die wahre Position befinden sollte. Ohne Messungen von weiteren Sensoren die R�ckschl�sse auf die Umgebung erlauben, w�rde die Positionssch�tzung mit der Zeit immer ungenauer. Mit Hilfe eines Messmodells l�sst sich beurteilen, ob eine Messung an einer Bestimmten Position wahrscheinlich erscheint, oder nicht. Dadurch l�sst sich die Wahrscheinlichkeitsverteilung der Position nach einer Bewegung durch eine Messung wieder eingrenzen. Auf den Bereich, in dem der Messwert des Sensors am Wahrscheinlichsten ist.
\item[Global Localization] ist das finden der Anfangsposition des Roboters unter allen M�glichen Posen die im Szenario vorkommen k�nnen. Im Vergleich zum \textit{Position Tracking}, bei dem es gen�gte die Unsicherheit um die gesch�tzte neue Pose zu ber�cksichtigen, umfasst hier der Raum m�glicher Posen ein erheblich gr��eres Volumen. Ein Ansatz w�re alle m�glichen Posen mit der selben Wahrscheinlichkeit anzunehmen, und mit den ersten Messungen und dem Messmodell diese einzugrenzen. Auf Bereiche in denen diese Messungen mit hoher Wahrscheinlichkeit auftritt. 
\item[Kidnapped Robot Problem] ist eine versch�rfte Variante des \textit{Global Localization} Problems. Dabei geht man davon aus, das sich der Roboter spontan an einem anderen Ort aufh�lt als vom Roboter angenommen. Nun m�sste die Lokalisation des Roboters wieder im gesamten m�glichen Raum erfolgen. Nur das der Roboter diesen Zustand nicht feststellen kann. 
\end{description}

Bayes Filter --> Markov Localization -->
\begin{description}
\item[Kalman Filter] EKF, UKF
\item[Partikel Filter] Sequenzielle Monte-Carlo-Methode
\item[Grid-Based-Filter] Histogram Filter (continous Space), discrete Bayes Filter (discrete Space)
\end{description}
\cite[Seite 115]{Hertzberg2012} \cite{Thrun2006} 

    \begin{table}[htdp]
      \centering
      \caption[Anzahl Probanden pro Gruppe bei $\alpha = 0,05$]{Anzahl Probanden \emph{pro Gruppe} nach Power und Effektst�rke bei $\alpha = 0,05$}
      \label{Sprungmarke2}
      \begin{tabular}{|r||r|r|r|r|r|r|r|r|r|r|}
        \hline
        \multicolumn{11}{|c|}{\cellcolor{tableBlue}$\mathbf{\alpha = 0.05}$} \\ \hline
        & \multicolumn{10}{c|}{\textbf{Effektst�rke}} \\
        \textbf{Power} & \textbf{0.10} & \textbf{0.15} & \textbf{0.20} & \textbf{0.25} & \textbf{0.30} & \textbf{0.40} & \textbf{0.50} & \textbf{0.60} & \textbf{0.70} & \textbf{0.80} \\ \hline\hline
        \textbf{.60}  &  977 &  434 & 244 & 156 & 109 &  61 &  39 &  27 & 20 & 15 \\ \hline
        \textbf{.70}  & 1230 &  547 & 308 & 197 & 137 &  77 &  49 &  34 & 25 & 19 \\ \hline
        \textbf{.80}  & 1568 &  697 & 392 & 251 & 174 &  98 &  63 &  44 & 32 & 25 \\ \hline
        \textbf{.90}  & 2100 &  933 & 525 & 336 & 233 & 131 &  84 &  58 & 43 & 33 \\ \hline
        \textbf{.95}  & 2592 & 1152 & 648 & 415 & 288 & 162 & 104 &  72 & 53 & 41 \\ \hline
        \textbf{.99}  & 3680 & 1636 & 920 & 589 & 409 & 230 & 147 & 102 & 75 & 58 \\ \hline
      \end{tabular}
    \end{table}

