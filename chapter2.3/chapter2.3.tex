\chapter{Lokalisierung mittels Bildverarbeitung}
\label{chap:lokalisierungmittelsbildverarbeitung}

\subsection{Partikel Filter}
\label{sub:PartikelFilter}
Der \textbf{Zustandsraum} der Partikel setzt sich aus der Position und der Pose des Roboters zusammen. Da er sich ausschlie�lich auf einer ebenen B�hne befindet, f�hrt dies zur Reduktion der Freiheitsgrade von sechs auf drei: X, Y und $\psi$. 

Das \textbf{Dynamikmodell} f�r den Roboter basiert auf der Odometrie. Als Input bekommt das Dynamik-Update die Inkremente des linken und rechten Rades. Es sind die relative Inkrementwerte, also seit dem letzten Dynamik-Update. Sie werden verwendet, um daraus eine Vorw�rtsfahrt und einen Drehwinkel zu berechnen. Zu diesen wird jeweils ein Gau�sches Rauschen addiert


Wieviele Partikel, und warum?
Wie sieht der Partikelraum aus?
Wie ist das Messmodell aufgebaut?
Besonderheiten im Messmodell: erst an 10 Punken wird ausgewertet, geringer Kontrast f�hrt zu Abwertung.
Wie wird initialisiert? 
Wie ist das Dynamikmodell aufgebaut?
Wie ist das Muster aufgebaut?

Wie wird die "gesch�tzte" Position berechnet?
