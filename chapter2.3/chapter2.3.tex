\chapter{Lokalisierung mittels Bildverarbeitung}
\label{chap:lokalisierungmittelsbildverarbeitung}

\section{Partikel Filter}
\label{sec:PartikelFilter}
Der \textbf{Zustandsraum} der Partikel setzt sich aus der Position und der Pose des Roboters zusammen. Da er sich ausschlie�lich auf einer ebenen B�hne befindet, f�hrt dies zur Reduktion der Freiheitsgrade von sechs auf drei: X, Y und $\psi$. 

Das \textbf{Dynamikmodell} f�r den Roboter basiert auf der Odometrie. Als Input bekommt das Dynamik-Update die Inkremente des linken und rechten Rades. Es sind absolute Inkrementwerte aus denen die Differenzen $ \Delta I_{r/l} $ zum letzten Update gebildet werden. Sie werden verwendet, um daraus eine Vorw�rtsfahrt $ \Delta s $ und einen Drehwinkel $ \Delta \psi $ zu berechnen:
\[\Delta s = \frac{\Delta I_r + \Delta I_l}{2}\cdot \underbrace{\frac{2\pi r}{g \cdot \gamma}}_{E}\]
\[\Delta \psi = \frac{\Delta I_r - \Delta I_l}{2}\cdot \frac{2\cdot E}{D}\]
mit Radabstand $ D $, Getriebe�bersetztung $ g $ und Geberaufl�sung $ \gamma $

Jedes Partikel berechnet daraus seinen neuen Zustand:
\[x_t = x_{t-1} + cos(\psi_{t-1} + \frac{\Delta \psi}{2})\cdot \Delta s\]
\[y_t = y_{t-1} + sin(\psi_{t-1} + \frac{\Delta \psi}{2})\cdot \Delta s\]
\[\psi_t = \psi_{t-1} +\Delta \psi\]
Dabei wird erst eine Drehung um $ \frac{\Delta \psi}{2} $ vollzogen, gefolgt von der Geradeausfahrt um $ \Delta s $ mit einer abschlie�enden Drehung um $ \frac{\Delta \psi}{2} $.
Damit der Partikel Filter funktioniert, muss er die Messunsicherheiten der Eingangswerte ber�cksichtigen. Dazu wird vor der Zustandsberechnung, zu $ \Delta s $ und $ \Delta \psi $ ein Gau�sches Rauschen addiert. Es ist proportional zu deren Betrag:
\[\Delta s_{err} = \Delta s \cdot \sigma_{s} \cdot RandomGaussian()\]
\[\Delta \psi_{err} = \Delta \psi \cdot \sigma_{\psi} \cdot RandomGaussian()\]
$ \sigma_{s} $ und $\sigma_{\psi}$ sind dabei ein Ma� daf�r wie breit die Streuung der Normalverteilung ist. Sie sind Parameter die auf den Anwendungsfall, nach St�rke des erwarteten Rauschens, eingestellt werden m�ssen. Dabei soll die Streuung der Partikel im Zustandsraum mindestens genau so gro� sein, wie die Streuung um den wahren Wert, verursacht durch Messunsicherheit der Sensoren. Wir das $\sigma$ zu klein gew�hlt, so kann es  passieren, dass die Verteilung der Partikel den wahren Zustand nicht mehr enth�lt. Somit gibt es bei einer Messung kein Partikel mehr, dessen Zustand diese als wahrscheinlich erscheinen l�sst. Damit folgen die Partikel im Zustandsraum einer falschen Sch�tzung, und die Messungen sind wertlos. Der Partikel Filter h�tte die Position verloren.

Setzt man das $\sigma$ gr��er an, so divergieren die Partikel mit jedem Dynamik-Update st�rker und der wahre Wert wird hoher Wahrscheinlichkeit von Partikeln abgedeckt, so dass bei einer Messung diese einen guten Score bekommen und durch ein Resampling sich die Partikel wieder um den wahren Wert konzentrieren. Allerdings ist bei zu gro�em $\sigma$ die Aussagekraft der Partikelverteilung sehr ungenau und es sind viele Partikel n�tig, um die n�tige Dichte im Zustandsraum zu gew�hrleisten. Dabei spielt es eine entscheidende Rolle, wie h�ufig Messungen erfolgen. Denn zwischen den Messungen muss sich der Filter auf das Dynamikmodell verlassen, und bei gro�em $\sigma$ ist die Sch�tzung nach wenigen Schritten bereits mit einer gro�en Unsicherheit verbunden.

Als Messungen werden die Bilder einer Kamera auf dem Roboter verwendet. Das \textbf{Messmodell} dahinter beruht auf dem Wissen um die Position von bestimmten Mustern in der Umgebung. Dies kann als Karte der Umgebung verstanden werden, anhand derer sich der Roboter orientieren muss. Es gibt drei verschiedene Muster, auf jeder Lichtwand eines. Die Muster werden in der obersten Zeile der Lichtwand angezeigt, um m�glichst selten verdeckt zu werden. Bei dem Messmodell gilt es nun zu pr�fen, ob ein Bild zu einer bestimmten Pose passt oder nicht. Daf�r k�nnte man in dem Bild nach den bekannten Mustern suchen, und sobald diese gefunden sind versuchen diese einer Pose zu zuordnen. Aber eine solche Mustersuche in einem Bild ist immer in verschiedene Schritte aufgeteilt, die auf einander aufbauen. Also z.B. Binarisierung �ber einen Schwellwert, Regionenbildung mit Charakterisierung und anschlie�ende Auswertung ausgew�hlter Regionen. Oder Kantenerkennung, Hough-Transformation und finden von parallelen kurzen Linien. Ein Problem daran ist, dass wenn in einem ersten Schritt z.B. ein Schwellwert falsch gew�hlt wurde, oder nur sehr schwache Kanten vorhanden sind, alle folgenden Schritte scheitern, weil ihre Vorbedingungen nicht ausreichend erf�llt werden. Aus diesem Grund wurde ein anderer Ansatz verfolgt, bei dem man nicht das Bild und die Informationen darin als Ausgangspunkt nimmt, sondern die Pose der Partikel und die Position der Muster im Raum. Dazu soll aus der Pose des Partikels die Pose der Kamera abgeleitet werden. Und anschlie�end die Position des Musters aus dem Raum in Pixelkoordinaten projiziert werden. Damit k�nnte man f�r jede beliebige Pose des Roboters sagen wo im Bild das Muster zu sehen sein m�sste und diese Bereiche mit dem erwarteten Muster vergleichen. Je besser der Bereich zu dem Muster passt, umso h�her wird der Score f�r die Partikel Bewertung.

{\color{red}
Wieviele Partikel, und warum?

Wie sieht der Partikelraum aus? 

Wie ist das Messmodell aufgebaut?

Besonderheiten im Messmodell: erst an 10 Punken wird ausgewertet, geringer Kontrast f�hrt zu Abwertung.

Wie wird initialisiert? 

Wie ist das Dynamikmodell aufgebaut?

Wie ist das Muster aufgebaut?

Wie wird die "gesch�tzte" Position berechnet?
}
