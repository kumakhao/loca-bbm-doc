\chapter{Lokalisierung mittels Bildverarbeitung}
\label{chap:lokalisierungmittelsbildverarbeitung}

\section{Partikel Filter}
\label{sec:PartikelFilter}
Der \textbf{Zustandsraum} der Partikel setzt sich aus der Position und der Pose des Roboters zusammen. Da er sich ausschlie�lich auf einer ebenen B�hne befindet, f�hrt dies zur Reduktion der Freiheitsgrade von sechs auf drei: X, Y und $\psi$. 

Das \textbf{Dynamikmodell} f�r den Roboter basiert auf der Odometrie. Als Input bekommt das Dynamik-Update die Inkremente des linken und rechten Rades. Es sind absolute Inkrementwerte aus denen die Differenzen $ \Delta I_{r/l} $ zum letzten Update gebildet werden. Sie werden verwendet, um daraus eine Vorw�rtsfahrt $ \Delta s $ und einen Drehwinkel $ \Delta \psi $ zu berechnen:
\[\Delta s = \frac{\Delta I_r + \Delta I_l}{2}\cdot \underbrace{\frac{2\pi r}{g \cdot \gamma}}_{E}\]
\[\Delta \psi = \frac{\Delta I_r - \Delta I_l}{2}\cdot \frac{2\cdot E}{D}\]
mit Radabstand $ D $, Getriebe�bersetztung $ g $ und Geberaufl�sung $ \gamma $

Jedes Partikel berechnet daraus seinen neuen Zustand:
\[x_t = x_{t-1} + cos(\psi_{t-1} + \frac{\Delta \psi + \psi_{err}}{2})\cdot (\Delta s + s_{err})\]
\[y_t = y_{t-1} + sin(\psi_{t-1} + \frac{\Delta \psi + \psi_{err}}{2})\cdot (\Delta s + s_{err})\]
\[\psi_t = \psi_{t-1} +\Delta \psi + \psi_{err}\]
Wieviele Partikel, und warum?
Wie sieht der Partikelraum aus?
Wie ist das Messmodell aufgebaut?
Besonderheiten im Messmodell: erst an 10 Punken wird ausgewertet, geringer Kontrast f�hrt zu Abwertung.
Wie wird initialisiert? 
Wie ist das Dynamikmodell aufgebaut?
Wie ist das Muster aufgebaut?

Wie wird die "gesch�tzte" Position berechnet?
